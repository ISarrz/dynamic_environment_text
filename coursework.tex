\documentclass[a4paper,12pt]{extarticle}
\usepackage{geometry}
\usepackage[T1]{fontenc}
\usepackage[utf8]{inputenc}
\usepackage[english,russian]{babel}
\usepackage{amsmath}
\usepackage{amsthm}
\usepackage{amssymb}
\usepackage{fancyhdr}
\usepackage{setspace}
\usepackage{graphicx}
\usepackage{colortbl}
\usepackage{tikz}
\usepackage{pgf}
\usepackage{subcaption}
\usepackage{listings}
\usepackage{indentfirst}
\usepackage{tabularx}
\usepackage{makecell}

\usepackage[
    backend=biber,
    style=numeric,
    maxbibnames=99
]{biblatex}
\addbibresource{refs.bib}
\usepackage[colorlinks,citecolor=blue,linkcolor=blue,bookmarks=false,hypertexnames=true, urlcolor=blue]{hyperref}

\usepackage{mathtools}
\usepackage{booktabs}
\usepackage[flushleft]{threeparttable}
\usepackage{tablefootnote}

\usepackage{chngcntr} % нумерация графиков и таблиц по секциям
\counterwithin{table}{section}
\counterwithin{figure}{section}

\graphicspath{{graphics/}}%путь к рисункам

\makeatletter
% \renewcommand{\@biblabel}[1]{#1.} % Заменяем библиографию с квадратных скобок на точку:
\makeatother

\geometry{left=2.5cm}% левое поле
\geometry{right=1.0cm}% правое поле
\geometry{top=2.0cm}% верхнее поле
\geometry{bottom=2.0cm}% нижнее поле
\setlength{\parindent}{1.25cm}
\renewcommand{\baselinestretch}{1.5} % междустрочный интервал


\newcommand{\bibref}[3]{\hyperlink{#1}{#2 (#3)}} % biblabel, authors, year
\addto\captionsrussian{\def\refname{Список литературы (или источников)}}

\renewcommand{\theenumi}{\arabic{enumi}}% Меняем везде перечисления на цифра.цифра
\renewcommand{\labelenumi}{\arabic{enumi}}% Меняем везде перечисления на цифра.цифра
\renewcommand{\theenumii}{.\arabic{enumii}}% Меняем везде перечисления на цифра.цифра
\renewcommand{\labelenumii}{\arabic{enumi}.\arabic{enumii}.}% Меняем везде перечисления на цифра.цифра
\renewcommand{\theenumiii}{.\arabic{enumiii}}% Меняем везде перечисления на цифра.цифра
\renewcommand{\labelenumiii}{\arabic{enumi}.\arabic{enumii}.\arabic{enumiii}.}% Меняем везде перечисления на цифра.цифра

\begin{document}
    \begin{titlepage}
\newpage

{\setstretch{1.0}
\begin{center}
ПРАВИТЕЛЬСТВО РОССИЙСКОЙ ФЕДЕРАЦИИ\\
ФГАОУ ВО НАЦИОНАЛЬНЫЙ ИССЛЕДОВАТЕЛЬСКИЙ УНИВЕРСИТЕТ\\
«ВЫСШАЯ ШКОЛА ЭКОНОМИКИ»
\\
\bigskip
Факультет компьютерных наук\\
Образовательная программа «Прикладная математика и информатика»
\end{center}
}

\vspace{10em}


\begin{center}
%Выберите какой у вас проект
{\bf Отчет об исследовательском проекте на тему:}\\
{\bf Автономная навигация мобильного робота с использованием Robotic Operating System: алгоритмы планирования и управления}\\
% строчка ниже нужна только при сдача плана КР, при финальной сдаче закомментируйте ее
(промежуточный, этап 1)
\end{center}

\vspace{2em}

{\bf Выполнил студент: \vspace{2mm}}
%{\bf Выполнили студенты: \vspace{2mm}}

{\setstretch{1.1}
\begin{tabular}{l@{\hskip 1.5cm}l}
группы \#БПМИ244, 2 курса & Злобин Герман Эдуардович \\
\end{tabular}}


\vspace{1em}
{\bf Принял руководитель проекта: \vspace{2mm}}

{\setstretch{1.1}
\begin{tabular}{l}
Дергачев Степан Алексеевич\\
Штатный преподаватель\\
Базовая кафедра Федерального исследовательского центра\\
«Информатика и управление» Российской академии наук
\end{tabular}}


\vspace{\fill}

\begin{center}
Москва 2026
\end{center}

\end{titlepage}

    \newpage
    \setcounter{page}{2}

    {
        \hypersetup{linkcolor=black}
        \tableofcontents
    }

    \newpage

    \newpage


    \section*{Аннотация}
    В работе рассматривается задача планирования движения автономного робота в динамической среде.
    Выполнен обзор и анализ современных алгоритмов глобального и локального планирования,
    рассмотрены их принципы работы, достоинства и ограничения.
    Проведено сравнительное исследование методов с точки зрения вычислительной эффективности и
    применимости в условиях изменяющейся среды.
    На основе анализа обоснован выбор подходов, наиболее подходящих для задач навигации мобильных роботов.


    \addcontentsline{toc}{section}{Аннотация}


    \section*{Ключевые слова}
    Навигация; планирование пути; динамическая среда; мобильные роботы; локальный и глобальный планировщик.
    \pagebreak


    \section{Введение}\label{sec:introduce}
    In progress\ldots

    \subsection{Описание области}\label{subsec:-description}

    \subsection{Постановка задачи}\label{subsec:-target}

    \subsection{Структура работы}\label{subsec:-structure}



    \newpage


    \section{Обзор литературы}
    Задаче планирования движения мобильных роботов в динамических средах
    посвящено большое количество научных исследований.
    В обзорных работах ~\cite{general, global_planing, tree_algorithms},
    предлагаются классификации методов планирования, анализируются их достоинства и ограничения.
    В данной главе рассматриваются основные классы алгоритмов глобального и
    локального планирования с акцентом на их применимость в условиях изменяющейся среды.

    \subsection{Обзор методов глобального планирования}
    Методы глобального планирования предназначены для построения маршрута от начальной точки к цели на основе априорной
    или частично известной карты среды.
    Как правило, они ориентированы на поиск оптимального или
    близкого к оптимальному пути и используются на верхнем уровне навигационной системы.

    \subsubsection{Алгоритмы поиска по графу}
    Алгоритмы поиска по графу относятся к классическим методам планирования и
    широко применяются в задачах навигации мобильных роботов~\cite{tree_algorithms}.
    Они предполагают дискретизацию пространства в виде графа или решётки и поиск пути между вершинами.

    \textbf{Алгоритм Дейкстры}~\cite{dijkstra} предназначен для поиска кратчайшего пути в графе с
    неотрицательными весами рёбер.
    Он гарантирует нахождение оптимального решения, однако требует обработки большого числа вершин,
    что приводит к высокой вычислительной сложности при увеличении размера карты.

    \textbf{Алгоритм A*}~\cite{Astar} является развитием метода Дейкстры и
    использует эвристическую функцию для ускорения поиска.
    При корректном выборе эвристики A* существенно снижает количество рассматриваемых вершин и
    демонстрирует высокую эффективность в двумерных и слаборазмерных средах.
    Тем не менее, в сложных и высокоразмерных пространствах его производительность снижается.

    В условиях динамической среды классические алгоритмы Дейкстры и A* требуют полного пересчёта маршрута
    при изменении информации о препятствиях, что ограничивает их применение в реальном времени.

    Для устранения данного недостатка были разработаны инкрементальные алгоритмы перепланирования.
    \textbf{Алгоритм D*}~\cite{Dstar} и его модификация \textbf{D* Lite}~\cite{dstarlite}
    позволяют обновлять ранее найденный путь при появлении новых данных об окружающей среде.
    \textbf{Алгоритм Lifelong Planning A* (LPA*)}~\cite{dstarlite} поддерживает повторное использование
    результатов предыдущего поиска. \textbf{Алгоритм ARA*}~\cite{anytime} реализует anytime-подход,
    обеспечивая быстрое получение приближённого решения с последующим улучшением.

    Таким образом, алгоритмы поиска по графу обеспечивают высокое качество планирования и
    предсказуемость поведения робота.
    Однако их применение в динамических средах ограничено вычислительной сложностью и зависимостью от точности карты.

    \subsubsection{Алгоритмы случайной выборки (Sampling-based)}
    Алгоритмы случайной выборки (sampling-based planners) предназначены для планирования движения в непрерывных и
    высокоразмерных пространствах конфигураций~\cite{rrt_review}.
    Они основаны на случайном семплировании допустимых состояний и построении графа или дерева достижимых конфигураций.

    \textbf{Алгоритм Rapidly-exploring Random Tree (RRT)}~\cite{rrt} осуществляет итеративное расширение дерева в
    направлении случайных точек пространства.
    Он быстро находит допустимые траектории в сложных средах и
    обладает хорошими исследовательскими свойствами.
    Однако получаемые пути, как правило, далеки от оптимальных.

    \textbf{Алгоритм RRT*}~\cite{rrt_star} является модификацией RRT и гарантирует асимптотическую оптимальность.
    По мере увеличения числа семплов качество найденного пути улучшается.
    Недостатком RRT* является высокая вычислительная сложность и медленная сходимость в средах с узкими проходами.

    \textbf{Алгоритм Batch Informed Trees (BIT*)}~\cite{bit} сочетает идеи эвристического поиска и семплирования.
    Он ограничивает область поиска с использованием эвристики, аналогичной A*, что повышает эффективность планирования.
    \textbf{Алгоритм RABIT*}~\cite{rabit} расширяет BIT* за счёт применения локальной оптимизации,
    что ускоряет сходимость к качественным решениям.

    Для задач с динамическими препятствиями предлагаются специализированные методы, например \textbf{Risk-DTRRT}~\cite{drrt},
    учитывающий вероятность столкновений и уровень риска при построении траектории.

    В целом семплинг-базированные алгоритмы хорошо масштабируются на пространства высокой размерности и
    не требуют полной дискретизации среды.
    Однако в динамических условиях они часто нуждаются в перестроении структуры поиска,
    что снижает их эффективность при быстром изменении обстановки.

    \subsubsection{Интеллектуальные бионические алгоритмы}
    Бионические и интеллектуальные алгоритмы основаны на моделировании коллективного поведения биологических систем и
    применяются для решения задач оптимизации~\cite{bio_review}.
    В планировании движения они используются для поиска приближённых оптимальных траекторий.

    \textbf{Генетические алгоритмы (GA)}~\cite{ga} представляют траекторию в виде хромосомы и используют операции скрещивания и
    мутации для поиска оптимального решения.
    Они обладают высокой гибкостью и способны адаптироваться к изменениям среды,
    однако требуют значительных вычислительных ресурсов и не гарантируют сходимости за ограниченное время.

    \textbf{Алгоритмы муравьиной колонии (ACO)}~\cite{aca} основаны на коллективном поиске пути
    с использованием виртуальных феромонов.
    Они эффективны в задачах поиска кратчайших маршрутов, но чувствительны к настройке параметров и
    могут медленно адаптироваться к динамическим изменениям.

    \textbf{Алгоритм искусственной пчелиной колонии (ABC)}~\cite{abc} и \textbf{алгоритм роя частиц (PSO)}~\cite{pso}
    применяются для оптимизации параметрических представлений траектории.
    Их преимуществами являются простота реализации и возможность параллельной обработки.
    Основным недостатком является зависимость качества решения от начальной инициализации и параметров.

    Бионические методы демонстрируют хорошую адаптивность,
    однако чаще используются в офлайн-планировании или в средах с медленно меняющейся динамикой.

    \begin{table}[ht]
        \caption{Сравнение алгоритмов глобального планирования}
        \label{table:global_algorithms}
        \footnotesize
        \centering
        \begin{tabularx}{\textwidth}{|c|c|c|c|c|}
            \hline
            \makecell[c]{Класс\\алгоритм} &
            \makecell[c]{Работа в\\динамической\\среде} &
            \makecell[c]{Вычислительная\\сложность} &
            \makecell[c]{Основные\\преимущества} &
            \makecell[c]{Основные\\недостатки}\\
            \hline
            \makecell[c]{Графовые\\\textbf{Dijkstra}} & Низкая & Высокая & \makecell[c]{Гарантированная\\оптимальность,\\простота} &
            \makecell[c]{Полный пересчёт\\при изменениях}\\
            \hline
            \makecell[c]{Графовые\\\textbf{A*}} & Низкая & Средняя &
            \makecell[c]{Быстрее Дейкстры,\\эвристический поиск} & \makecell[c]{Зависит от эвристики}\\
            \hline
            \makecell[c]{Графовые\\\textbf{D*}} & Высокая & Средняя &
            \makecell[c]{Инкрементальное\\перепланирование} & \makecell[c]{Сложная реализация}\\
            \hline
            \makecell[c]{Графовые\\\textbf{D* Lite}} & Высокая & Средняя &
            \makecell[c]{Эффективное\\обновление пути} & \makecell[c]{Требует памяти}\\
            \hline
            \makecell[c]{Графовые\\\textbf{LPA*}} &
            Высокая & Средняя &
            \makecell[c]{Повторное\\использование\\поиска} & \makecell[c]{Медленнее\\D* Lite}\\
            \hline
            \makecell[c]{Графовые\\\textbf{ARA*}} &
            Средняя & Средняя &
            \makecell[c]{Быстрое\\первое решение} & \makecell[c]{Временно\\субоптимален}\\
            \hline
            \makecell[c]{Семплинг\\\textbf{RRT}} &
            Средняя & Низкая &
            \makecell[c]{Быстро находит путь} & \makecell[c]{Плохое качество пути}\\
            \hline
            \makecell[c]{Семплинг\\\textbf{RRT*}} &
            Средняя & Высокая &
            \makecell[c]{Оптимальные\\траектории} & \makecell[c]{Медленная сходимость}\\
            \hline
            \makecell[c]{Семплинг\\\textbf{BIT*}} &
            Средняя & Средняя &
            \makecell[c]{Эвристическое\\ускорение} & \makecell[c]{Сложная настройка}\\
            \hline
            \makecell[c]{Семплинг\\\textbf{RABIT*}} &
            Средняя & Средняя &
            \makecell[c]{Быстрая оптимизация} & \makecell[c]{Зависит от локального\\оптимизатора}\\
            \hline
            \makecell[c]{Семплинг\\\textbf{Risk-DTRRT}} &
            Высокая & Высокая &
            \makecell[c]{Учет неопределённости} & \makecell[c]{Сложность модели}\\
            \hline
            \makecell[c]{Бионические\\\textbf{GA}} &
            Средняя & Высокая &
            \makecell[c]{Гибкость, адаптация} & \makecell[c]{Медленная сходимость}\\
            \hline
            \makecell[c]{Бионические\\\textbf{ACO}} &
            Средняя & Средняя &
            \makecell[c]{Распределённый\\поиск} & \makecell[c]{Чувствителен к параметрам}\\
            \hline
            \makecell[c]{Бионические\\\textbf{ABC}} &
            Средняя & Средняя &
            \makecell[c]{Простота реализации} & \makecell[c]{Нестабильность}\\
            \hline
            \makecell[c]{Бионические\\\textbf{PSO}} &
            Средняя & Средняя &
            \makecell[c]{Быстро сходится} & \makecell[c]{Локальные минимумы}\\
            \hline
        \end{tabularx}

    \end{table}
    %\makecell[c]{}

    \newpage
    \\ \quad
    \newpage

    \subsection{Обзор методов локального планирования}
    Методы локального планирования предназначены для формирования управляющих воздействий в реальном времени на основе
    текущего состояния робота и информации о ближайшем окружении.
    В отличие от глобальных планировщиков, которые строят маршрут на всей карте,
    локальные методы работают в ограниченной области и ориентированы прежде всего на безопасное движение,
    обход препятствий и следование заданному глобальному пути.
    Это делает их особенно важными в динамических средах,
    где присутствуют движущиеся объекты и неопределённость в данных сенсоров.
    Как правило, локальные планировщики используются совместно с глобальными: глобальный уровень задаёт опорный маршрут,
    а локальный корректирует движение робота с учётом текущей ситуации.

    \subsubsection{Классические контроллеры}
    Классические контроллеры применяются в первую очередь для задачи слежения за заданной траекторией и
    стабилизации движения робота.
    Они не выполняют планирование траектории в строгом смысле,
    однако широко используются как нижний исполнительный уровень в навигационных системах.

    \textbf{PID-контроллер} является одним из наиболее простых и распространённых регуляторов.
    Он формирует управляющее воздействие на основе текущей ошибки, её интегральной и дифференциальной составляющих.
    В задачах мобильной робототехники PID применяется для управления линейной и угловой скоростью робота
    с целью достижения заданной позиции или следования траектории~\cite{pid}.
    Основными преимуществами PID-регулятора являются простота реализации,
    низкие вычислительные затраты и высокая частота работы.
    К недостаткам относится отсутствие учёта динамики окружающей среды и
    невозможность самостоятельного избегания препятствий,
    что ограничивает его применение только исполнительным уровнем.

    Основная цель — заставить робота перемещаться в заданную позицию с помощью PID‑регулятора,
    настроенного на управление линейным и угловым положением.
    PID‑контроллер используется для регулирования скорости и ориентации робота так,
    чтобы он достигал требуемой позиции.

    \textbf{Линейно-квадратичный регулятор (LQR)} основан на минимизации квадратичного функционала качества,
    учитывающего отклонение состояния системы от заданного и величину управляющего воздействия.
    В работах~\cite{lqr} и~\cite{lqr2} показано,
    что LQR позволяет обеспечить более устойчивое и оптимальное движение мобильного робота по сравнению с PID,
    особенно при задаче слежения за траекторией.
    LQR учитывает модель динамики системы,
    что повышает точность управления.
    Его основным недостатком является необходимость линеаризации модели и чувствительность к ошибкам параметров.

    \textbf{Комбинированные схемы PID+LQR} используются для объединения простоты PID и оптимальных свойств LQR\@.
    В работе~\cite{pid_lqr} показано, что такой подход позволяет повысить устойчивость системы,
    улучшить качество позиционирования и подавить колебания.
    Однако гибридные контроллеры требуют более сложной настройки и точного согласования параметров.

    В целом классические контроллеры не являются полноценными локальными планировщиками,
    но выполняют важную роль исполнительного уровня, обеспечивая реализацию траекторий,
    полученных более высокими уровнями навигационной системы.

    \subsubsection{Геометрические или реактивные методы}
    Геометрические или реактивные методы локального планирования принимают решения непосредственно
    на основе текущих сенсорных данных и геометрии окружающей среды.
    Они не строят долгосрочные траектории, а выбирают допустимое направление или скорость движения,
    обеспечивающие безопасность в ближайшем будущем.

    \textbf{Метод Dynamic Window Approach (DWA)} был предложен в~\cite{dwa_origin}.
    Он рассматривает пространство допустимых линейных и угловых скоростей робота и выбирает те из них,
    которые достижимы с учётом динамических ограничений и не приводят к столкновениям.
    Далее используется целевая функция, учитывающая приближение к цели,
    скорость движения и расстояние до препятствий.
    В работе~\cite{dwa_new} предложена модификация DWA, ориентированная на динамические препятствия и групповые движения роботов.
    Основным достоинством DWA является высокая скорость работы и практическая применимость,
    однако метод подвержен застреванию в локальных минимумах.

    \textbf{Метод Velocity Obstacles (VO)} был предложен в~\cite{vo_origin} и основан на построении множества скоростей,
    которые приведут к столкновению с препятствием в будущем.
    Эти скорости исключаются из допустимого пространства управления.
    Такой подход позволяет учитывать движение препятствий и прогнозировать возможные столкновения.
    В работе~\cite{vo_new} предложено расширение VO, учитывающее как положение, так и скорости объектов.
    Преимуществом метода является корректная работа в динамических средах, однако он чувствителен к шуму в измерениях.

    \textbf{Метод искусственных потенциальных полей (APF)} был предложен Хатибом в~\cite{apf_orig}.
    Цель моделируется как источник притяжения, а препятствия — как источники отталкивания,
    и робот движется по направлению результирующего градиента.
    В~\cite{apf_new} предложена модификация APF для задач навигации в неизвестной среде с динамическими препятствиями.
    Основными преимуществами APF являются простота и высокая вычислительная эффективность.
    Существенным недостатком является наличие локальных минимумов, из-за которых робот может застревать.

    \textbf{Метод Vector Field Histogram (VFH)} был представлен в~\cite{vfh_orig} и
    основан на построении гистограммы плотности препятствий по данным сенсоров.
    На её основе выбирается направление движения с минимальной опасностью столкновения.
    В работе~\cite{vfh_new} предложены модификации VFH,
    улучшающие точность построения гистограмм и учитывающие динамические препятствия.
    Метод хорошо работает в реальном времени, однако не обеспечивает глобальной оптимальности траектории.

    Геометрические методы отличаются высокой вычислительной эффективностью и
    широко применяются в реальных робототехнических системах.
    Их основным недостатком является локальный характер принятия решений,
    что может приводить к неоптимальному поведению в сложных средах.

    \subsubsection{Оптимизационные или предсказательные методы}

    Оптимизационные методы формулируют задачу локального планирования как задачу оптимального управления,
    в которой необходимо минимизировать функционал качества при наличии динамических и кинематических ограничений.

    \textbf{Model Predictive Control (MPC)} является одним из наиболее распространённых методов этого класса.
    В обзоре~\cite{mpc} показано, что MPC позволяет учитывать динамику робота,
    ограничения на управление и наличие препятствий.
    Алгоритм на каждом шаге решает задачу оптимизации на конечном горизонте прогнозирования,
    обеспечивая высокое качество траекторий.
    Основным недостатком является высокая вычислительная сложность.

    \textbf{Model Predictive Path Integral (MPPI)} представляет собой стохастический вариант MPC,
    основанный на выборке большого числа возможных траекторий и оценке их стоимости~\cite{mppis}.
    В работе~\cite{mppi} MPPI успешно применён для задачи агрессивного автономного вождения.
    Преимуществами MPPI являются возможность работы с нелинейной динамикой и с
    ложными функциями стоимости.
    Недостатком является необходимость больших вычислительных ресурсов.

    \textbf{Алгоритм CCS-MPPI}, предложенный в~\cite{ccsmppi}, объединяет MPPI с методом Covariance Steering и
    позволяет управлять не только средним значением состояния, но и его ковариацией,
    обеспечивая вероятностные гарантии соблюдения ограничений.
    Этот подход особенно эффективен в стохастических средах, но сложен в реализации.

    \textbf{Метод RMPPI}, предложенный в~\cite{rmppi}, направлен на повышение робастности MPPI к ошибкам модели и внешним возмущениям.
    Он сочетает MPPI с трекинг-контроллером и механизмами безопасного распространения номинальной траектории.

    \textbf{Метод Adaptive MPPI}
    предложен в работе~\cite{adaptive_mppi},
    в нём параметры модели и контроллера автоматически настраиваются под текущие условия среды.
    Это позволяет повысить устойчивость алгоритма при неточной модели динамики.

    \textbf{Метод MPPI with Costmap},
    представленный в работе~\cite{mppi_with_map}, использует локальную карту затрат (cost map),
    формируемую на основе априорной карты и данных с камеры,
    что позволяет интегрировать глобальную информацию в локальное планирование.

    Оптимизационные методы обеспечивают наивысшее качество траекторий и наилучшее учёт динамики робота и среды,
    однако их применение ограничено высокими вычислительными затратами и сложностью реализации.

    \begin{table}[ht]
        \caption{Сравнение алгоритмов локального планирования}
        \label{table:local_algorithms}
        \footnotesize
        \centering
        \begin{tabularx}{\textwidth}{|c|c|c|c|c|}
            \hline
            \makecell[c]{Класс\\алгоритм} &
            \makecell[c]{Работа в\\динамической\\среде} &
            \makecell[c]{Вычислительная\\сложность} &
            \makecell[c]{Основные\\преимущества} &
            \makecell[c]{Основные\\недостатки}\\
            \hline
            \makecell[c]{Классические\\контроллеры\\\textbf{PID}} & Низкая & Низкая & \makecell[c]{Простота, \\высокая частота\\управления} &
            \makecell[c]{Нет планирования,\\нет обхода препятствий}\\
            \hline
            \makecell[c]{Классические\\контроллеры\\\textbf{LQR}} & Средняя & Средняя & \makecell[c]{Оптимальность, \\устойчивость} &
            \makecell[c]{Требует линеаризации}\\
            \hline
            \makecell[c]{Классические\\контроллеры\\\textbf{PID+LQR}} & Средняя & Средняя & \makecell[c]{Повышенная \\устойчивость и \\точность} &
            \makecell[c]{Сложность настройки}\\
            \hline
            \makecell[c]{Геометрические\\\textbf{DWA}} & Высокая & Низкая & \makecell[c]{Реактивность, \\практическая\\применимость} &
            \makecell[c]{Локальные минимумы}\\
            \hline
            \makecell[c]{Геометрические\\\textbf{VO}} & Высокая & \makecell[c]{Низкая–\\средняя} & \makecell[c]{Учёт движущихся\\препятствий} &
            \makecell[c]{Чувствительность\\к шуму}\\
            \hline
            \makecell[c]{Геометрические\\\textbf{APF}} & Средняя & Очень низкая & \makecell[c]{Простота,\\высокая скорость} &
            \makecell[c]{Локальные минимумы}\\
            \hline
            \makecell[c]{Геометрические\\\textbf{VFH}} & Высокая & Низкая & \makecell[c]{Эффективная\\работа с сенсорами} &
            \makecell[c]{Нет глобальной\\оптимальности}\\
            \hline
            \makecell[c]{Оптимизационные\\\textbf{MPC}} & Высокая & Высокая & \makecell[c]{Учёт ограничений\\и динамики} &
            \makecell[c]{Большие\\вычислительные затраты}\\
            \hline
            \makecell[c]{Оптимизационные\\\textbf{MPPI}} & Высокая & Высокая & \makecell[c]{Качественные\\траектории} &
            \makecell[c]{Требует параллелизма}\\
            \hline
            \makecell[c]{Оптимизационные\\\textbf{CCS-MPPI}} & \makecell[c]{Очень\\высокая}  & \makecell[c]{Очень\\высокая}  & \makecell[c]{Вероятностные\\гарантии\\безопасности} &
            \makecell[c]{Сложная реализация}\\
            \hline
            \makecell[c]{Оптимизационные\\\textbf{RMPPI}} &  \makecell[c]{Очень\\высокая} & \makecell[c]{Очень\\высокая}  & \makecell[c]{Робастность к\\ошибкам модели} &
            \makecell[c]{Высокая\\вычислительная цена}\\
            \hline
            \makecell[c]{Оптимизационные\\\textbf{Adaptive MPPI}} & \makecell[c]{Очень\\высокая}  & \makecell[c]{Очень\\высокая}  & \makecell[c]{Адаптация к\\неопределённости} &
            \makecell[c]{Сложность настройки}\\
            \hline
            \makecell[c]{Оптимизационные\\\textbf{MPPI + Costmap}} & \makecell[c]{Очень\\высокая}  & \makecell[c]{Очень\\высокая}  & \makecell[c]{Интеграция\\глобальной\\информации} &
            \makecell[c]{Большая\\вычислительная нагрузка}\\
            \hline


        \end{tabularx}

    \end{table}
    %\makecell[c]{}

    \newpage \quad \newpage
    \subsection{Вывод по главе}
    В данной главе был выполнен обзор основных подходов к планированию движения мобильных роботов,
    включающий методы глобального и локального планирования.
    Рассмотрены алгоритмы поиска по графу, семплинг-базированные и бионические методы глобального планирования,
    а также классические контроллеры, геометрические и оптимизационные методы локального уровня.
    Для каждого класса алгоритмов были проанализированы принципы работы, области применения, преимущества и ограничения.

    Показано, что методы глобального планирования позволяют формировать маршрут движения робота с учётом структуры среды и
    обеспечивают достижение целевой точки,
    однако в динамических условиях их эффективность ограничивается необходимостью частого перепланирования и
    высокой вычислительной сложностью. Алгоритмы поиска по графу обеспечивают высокую предсказуемость и качество решений,
    семплинг-базированные методы лучше масштабируются на сложные и высокоразмерные пространства,
    а бионические алгоритмы обладают высокой гибкостью, но требуют значительных вычислительных ресурсов
    и тщательной настройки.

    Методы локального планирования, в свою очередь,
    обеспечивают реактивное управление движением робота и его безопасное взаимодействие с динамической средой.
    Геометрические методы отличаются высокой скоростью работы и простотой реализации,
    оптимизационные и предсказательные методы позволяют получать траектории высокого качества с учётом динамики и
    ограничений, а классические контроллеры играют роль исполнительного уровня,
    реализующего полученные управляющие воздействия.


    Таким образом, анализ показал, что ни один из рассмотренных подходов не является универсальным и
    полностью самостоятельным решением задачи навигации в динамической среде.
    Наиболее перспективным и практически оправданным является использование гибридных навигационных систем,
    в которых глобальный планировщик формирует опорный маршрут,
    локальный планировщик обеспечивает адаптацию к текущей обстановке и избегание препятствий,
    а классические контроллеры реализуют устойчивое и точное управление движением робота.
    Такой многоуровневый подход позволяет совместить преимущества различных методов и
    повысить надёжность и эффективность автономной навигации.


%    \newpage
%    Это я сохранил всю литературу которую использовал на всякий случай
%    \cite{abc}
%    \cite{abc_aply}
%    \cite{aca}
%    \cite{aca_aply}
%    \cite{anytime}
%    \cite{aply_of_dijkstra}
%    \cite{Astar}
%    \cite{Astar_aply}
%    \cite{bfs}
%    \cite{bio_review}
%    \cite{bit}
%    \cite{dfs_aply}
%    \cite{dijkstra}
%    \cite{drrt}
%    \cite{Dstar}
%    \cite{dstarlite}
%    \cite{ga}
%    \cite{ga_aply}
%    \cite{general}
%    \cite{global_planing}
%    \cite{pso}
%    \cite{pso_aply}
%    \cite{rabit}
%    \cite{ros}
%    \cite{rrt}
%    \cite{rrt_review}
%    \cite{rrt_star}
%    \cite{rrt_star_apply}
%    \cite{tree_algorithms}


    \newpage
    \printbibliography[heading=bibintoc]

\end{document}
