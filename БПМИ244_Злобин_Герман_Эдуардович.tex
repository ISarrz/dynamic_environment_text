\documentclass[a4paper,12pt]{extarticle}

% -----------------------------
% Геометрия страницы
% -----------------------------
\usepackage{geometry}
\geometry{
    left=2.5cm,
    right=1.0cm,
    top=2.0cm,
    bottom=2.0cm
}

% -----------------------------
% Кодировки и языки
% -----------------------------
\usepackage[T2A]{fontenc}     % важно для русского в pdflatex
\usepackage[utf8]{inputenc}
\usepackage[english,russian]{babel}

% Чтобы кавычки и цитаты в biblatex были корректные
\usepackage{csquotes}

% -----------------------------
% Математика
% -----------------------------
\usepackage{amsmath}
\usepackage{amsthm}
\usepackage{amssymb}
\usepackage{mathtools}

% -----------------------------
% Графика и цвет
% -----------------------------
\usepackage{graphicx}
\graphicspath{{graphics/}}
\usepackage{xcolor}
\usepackage{colortbl}

% -----------------------------
% Таблицы
% -----------------------------
\usepackage{booktabs}
\usepackage{tabularx}
\usepackage{makecell}
\usepackage[flushleft]{threeparttable}
\usepackage{tablefootnote}

% -----------------------------
% Рисунки и подписи
% -----------------------------
\usepackage{tikz}
\usepackage{pgf}
\usepackage{subcaption}

% -----------------------------
% Листинги кода
% -----------------------------
\usepackage{listings}

% -----------------------------
% Прочее оформление
% -----------------------------
\usepackage{indentfirst}
\usepackage{setspace}
\usepackage{fancyhdr}

\setlength{\parindent}{1.25cm}
\onehalfspacing

% -----------------------------
% Нумерация таблиц и рисунков по секциям
% -----------------------------
\usepackage{chngcntr}
\counterwithin{table}{section}
\counterwithin{figure}{section}

% -----------------------------
% Библиография: biblatex + biber
% -----------------------------
\usepackage[
    backend=biber,
    style=numeric,
    sorting=none,
    maxbibnames=99
]{biblatex}

\addbibresource{refs.bib}

% Переименовать заголовок списка литературы
\addto\captionsrussian{\def\refname{Список литературы}}

% -----------------------------
% Hyperref — почти всегда в конце
% -----------------------------
\usepackage[
    colorlinks=true,
    citecolor=blue,
    linkcolor=blue,
    urlcolor=blue,
    bookmarks=false,
    hypertexnames=true
]{hyperref}

% -----------------------------
% Ваши команды (если нужны)
% -----------------------------
\newcommand{\bibref}[3]{\hyperlink{#1}{#2 (#3)}}

% -----------------------------
% Нумерация списков (как у вас)
% -----------------------------
\renewcommand{\theenumi}{\arabic{enumi}}
\renewcommand{\labelenumi}{\arabic{enumi}}
\renewcommand{\theenumii}{.\arabic{enumii}}
\renewcommand{\labelenumii}{\arabic{enumi}.\arabic{enumii}.}
\renewcommand{\theenumiii}{.\arabic{enumiii}}
\renewcommand{\labelenumiii}{\arabic{enumi}.\arabic{enumii}.\arabic{enumiii}.}

% -----------------------------
% Конец преамбулы
% -----------------------------




\begin{document}
    \begin{titlepage}
\newpage

{\setstretch{1.0}
\begin{center}
ПРАВИТЕЛЬСТВО РОССИЙСКОЙ ФЕДЕРАЦИИ\\
ФГАОУ ВО НАЦИОНАЛЬНЫЙ ИССЛЕДОВАТЕЛЬСКИЙ УНИВЕРСИТЕТ\\
«ВЫСШАЯ ШКОЛА ЭКОНОМИКИ»
\\
\bigskip
Факультет компьютерных наук\\
Образовательная программа «Прикладная математика и информатика»
\end{center}
}

\vspace{10em}


\begin{center}
%Выберите какой у вас проект
{\bf Отчет об исследовательском проекте на тему:}\\
{\bf Автономная навигация мобильного робота с использованием Robotic Operating System: алгоритмы планирования и управления}\\
% строчка ниже нужна только при сдача плана КР, при финальной сдаче закомментируйте ее
(промежуточный, этап 1)
\end{center}

\vspace{2em}

{\bf Выполнил студент: \vspace{2mm}}
%{\bf Выполнили студенты: \vspace{2mm}}

{\setstretch{1.1}
\begin{tabular}{l@{\hskip 1.5cm}l}
группы \#БПМИ244, 2 курса & Злобин Герман Эдуардович \\
\end{tabular}}


\vspace{1em}
{\bf Принял руководитель проекта: \vspace{2mm}}

{\setstretch{1.1}
\begin{tabular}{l}
Дергачев Степан Алексеевич\\
Штатный преподаватель\\
Базовая кафедра Федерального исследовательского центра\\
«Информатика и управление» Российской академии наук
\end{tabular}}


\vspace{\fill}

\begin{center}
Москва 2026
\end{center}

\end{titlepage}

    \newpage
    \setcounter{page}{2}

    {
        \hypersetup{linkcolor=black}
        \tableofcontents
    }

    \newpage

    \newpage


    \section*{Аннотация}

    В работе рассматривается задача планирования движения автономного робота в динамической среде. Выполнен обзор современных алгоритмов, обосновано их разделение на глобальные и локальные уровни. Проведено сравнительное исследование методов, на основе которого обоснован выбор алгоритма $D^*$ Lite. Разработана методика оценки эффективности $D^*$ Lite и $A^*$ по критериям времени и стабильности перепланирования. На основе 100 тестов установлено, что $D^*$ Lite сокращает время обновления пути в 29,7 раза, обеспечивая высокую предсказуемость работы системы в режиме реального времени.


    \section*{Ключевые слова}
    Навигация; планирование пути; динамическая среда; мобильные роботы; D* Lite; A*; локальный и глобальный планировщик.
    \pagebreak

\section{Введение}\label{sec:introduce}

Современная мобильная робототехника активно внедряется в логистику и автоматизацию, где ключевой задачей является обеспечение автономной навигации. Способность робота безопасно и эффективно прокладывать маршрут напрямую определяет его функциональность в реальных условиях.

\subsection{Описание области и постановка задачи}\label{subsec:-description}

Реальные среды эксплуатации (склады, цеха, городская застройка) являются динамическими: положение препятствий и объектов в них заранее неизвестно или постоянно меняется. В таких условиях классические алгоритмы планирования, требующие полной перестройки маршрута при каждом изменении, часто оказываются избыточными и неэффективными.

Целью данной работы является сравнительный анализ алгоритмов планирования пути в динамических средах. Основное внимание уделено переходу от классического поиска (A*) к инкрементальным методам ($D^*$ Lite), позволяющим обновлять маршрут без полного пересчета графа состояний.



Для достижения цели решаются задачи обзора методов навигации, обоснования архитектуры системы, а также программной реализации модели для замера времени перепланирования и вычислительных затрат выбранных алгоритмов.

\subsection{Структура работы}\label{subsec:-structure}

Работа включает введение, три главы, заключение и список литературы. Первая глава посвящена анализу предметной области и требованиям к навигации. Во второй главе проводится классификация алгоритмов и обоснование выбора $D^*$ Lite. Третья глава содержит описание методики эксперимента и анализ результатов сравнения $A^*$ и $D^*$ Lite по критериям скорости и стабильности вычислений на крупномасштабных картах.
    \pagebreak

    \section{Обзор литературы}
    В современных системах управления автономными роботами задачу навигации принято разделять на
    два иерархических уровня: глобальное и локальное планирование.
    Такое разделение обусловлено необходимостью баланса между вычислительной сложностью и
    скоростью реакции на изменения среды.

    Глобальное планирование отвечает за построение оптимального маршрута на всей известной карте.
    Однако расчет такого пути требует значительных ресурсов и не может выполняться с высокой частотой.
    Локальное планирование, напротив, работает в ограниченном «окне» видимости сенсоров,
    обеспечивая безопасный объезд внезапно возникших препятствий и у
    держание робота на глобальной траектории в режиме реального времени.
    Совместное использование этих уровней позволяет роботу сохранять целенаправленность движения,
    не теряя при этом реактивности.

    Задаче планирования движения мобильных роботов в динамических средах
    посвящено большое количество научных исследований.
    В обзорных работах ~\cite{general, global_planing, tree_algorithms},
    предлагаются классификации методов планирования, анализируются их достоинства и ограничения.
    В данной главе рассматриваются основные классы алгоритмов глобального и
    локального планирования с акцентом на их применимость в условиях изменяющейся среды.

    \subsection{Обзор методов глобального планирования}
    Методы глобального планирования предназначены для построения маршрута от начальной точки к цели на основе априорной
    или частично известной карты среды.
    Как правило, они ориентированы на поиск оптимального или
    близкого к оптимальному пути и используются на верхнем уровне навигационной системы.

    \subsubsection{Алгоритмы поиска по графу}
    Алгоритмы поиска по графу относятся к классическим методам планирования и
    широко применяются в задачах навигации мобильных роботов~\cite{tree_algorithms}.
    Они предполагают дискретизацию пространства в виде графа или решётки и поиск пути между вершинами.

    \textbf{Алгоритм Дейкстры}~\cite{dijkstra} предназначен для поиска кратчайшего пути в графе с
    неотрицательными весами рёбер.
    Он гарантирует нахождение оптимального решения, однако требует обработки большого числа вершин,
    что приводит к высокой вычислительной сложности при увеличении размера карты.

    \textbf{Алгоритм A*}~\cite{Astar} является развитием метода Дейкстры и
    использует эвристическую функцию для ускорения поиска.
    При корректном выборе эвристики A* существенно снижает количество рассматриваемых вершин и
    демонстрирует высокую эффективность в двумерных и слаборазмерных средах.
    Тем не менее, в сложных и высокоразмерных пространствах его производительность снижается.

    В условиях динамической среды классические алгоритмы Дейкстры и A* требуют полного пересчёта маршрута
    при изменении информации о препятствиях, что ограничивает их применение в реальном времени.

    Для устранения данного недостатка были разработаны инкрементальные алгоритмы перепланирования.
    \textbf{Алгоритм D*}~\cite{Dstar} и его модификация \textbf{D* Lite}~\cite{dstarlite}
    позволяют обновлять ранее найденный путь при появлении новых данных об окружающей среде.
    \textbf{Алгоритм Lifelong Planning A* (LPA*)}~\cite{dstarlite} поддерживает повторное использование
    результатов предыдущего поиска. \textbf{Алгоритм ARA*}~\cite{anytime} реализует anytime-подход,
    обеспечивая быстрое получение приближённого решения с последующим улучшением.

    \subsubsection{Алгоритмы случайной выборки (Sampling-based)}
    Алгоритмы случайной выборки (sampling-based planners) предназначены для планирования движения в непрерывных и
    высокоразмерных пространствах конфигураций ~\cite{rrtreview}.
    Они основаны на случайном семплировании допустимых состояний и построении графа или дерева достижимых конфигураций.

    \textbf{Алгоритм Rapidly-exploring Random Tree (RRT)}~\cite{rrt} осуществляет итеративное расширение дерева в
    направлении случайных точек пространства.
    Он быстро находит допустимые траектории в сложных средах и
    обладает хорошими исследовательскими свойствами.
    Однако получаемые пути, как правило, далеки от оптимальных.

    \textbf{Алгоритм RRT*}~\cite{rrt_star} является модификацией RRT и гарантирует асимптотическую оптимальность.
    По мере увеличения числа семплов качество найденного пути улучшается.
    Недостатком RRT* является высокая вычислительная сложность и медленная сходимость в средах с узкими проходами.

    \textbf{Алгоритм Batch Informed Trees (BIT*)}~\cite{bit} сочетает идеи эвристического поиска и семплирования.
    Он ограничивает область поиска с использованием эвристики, аналогичной A*, что повышает эффективность планирования.
    \textbf{Алгоритм RABIT*}~\cite{rabit} расширяет BIT* за счёт применения локальной оптимизации,
    что ускоряет сходимость к качественным решениям.

    Для задач с динамическими препятствиями предлагаются специализированные методы, например \textbf{Risk-DTRRT}~\cite{drrt},
    учитывающий вероятность столкновений и уровень риска при построении траектории.

    \subsubsection{Интеллектуальные бионические алгоритмы}
    Бионические и интеллектуальные алгоритмы основаны на моделировании коллективного поведения биологических систем и
    применяются для решения задач оптимизации~\cite{bio_review}.
    В планировании движения они используются для поиска приближённых оптимальных траекторий.

    \textbf{Генетические алгоритмы (GA)}~\cite{ga} представляют траекторию в виде хромосомы и используют операции скрещивания и
    мутации для поиска оптимального решения.
    Они обладают высокой гибкостью и способны адаптироваться к изменениям среды,
    однако требуют значительных вычислительных ресурсов и не гарантируют сходимости за ограниченное время.

    \textbf{Алгоритмы муравьиной колонии (ACO)}~\cite{aca} основаны на коллективном поиске пути
    с использованием виртуальных феромонов.
    Они эффективны в задачах поиска кратчайших маршрутов, но чувствительны к настройке параметров и
    могут медленно адаптироваться к динамическим изменениям.

    \textbf{Алгоритм искусственной пчелиной колонии (ABC)}~\cite{abc} и \textbf{алгоритм роя частиц (PSO)}~\cite{pso}
    применяются для оптимизации параметрических представлений траектории.
    Их преимуществами являются простота реализации и возможность параллельной обработки.
    Основным недостатком является зависимость качества решения от начальной инициализации и параметров.


    \pagebreak

    \subsection{Обзор методов локального планирования}
    Методы локального планирования предназначены для формирования управляющих воздействий в реальном времени на основе
    текущего состояния робота и информации о ближайшем окружении.
    В отличие от глобальных планировщиков, которые строят маршрут на всей карте,
    локальные методы работают в ограниченной области и ориентированы прежде всего на безопасное движение,
    обход препятствий и следование заданному глобальному пути.
    Это делает их особенно важными в динамических средах,
    где присутствуют движущиеся объекты и неопределённость в данных сенсоров.
    Как правило, локальные планировщики используются совместно с глобальными: глобальный уровень задаёт опорный маршрут,
    а локальный корректирует движение робота с учётом текущей ситуации.

    \subsubsection{Классические контроллеры}
    Классические контроллеры применяются в первую очередь для задачи слежения за заданной траекторией и
    стабилизации движения робота.
    Они не выполняют планирование траектории в строгом смысле,
    однако широко используются как нижний исполнительный уровень в навигационных системах.

    \textbf{PID-контроллер} является одним из наиболее простых и распространённых регуляторов.
    Он формирует управляющее воздействие на основе текущей ошибки, её интегральной и дифференциальной составляющих.
    В задачах мобильной робототехники PID применяется для управления линейной и угловой скоростью робота
    с целью достижения заданной позиции или следования траектории~\cite{pid}.
    Основными преимуществами PID-регулятора являются простота реализации,
    низкие вычислительные затраты и высокая частота работы.
    К недостаткам относится отсутствие учёта динамики окружающей среды и
    невозможность самостоятельного избегания препятствий,
    что ограничивает его применение только исполнительным уровнем.

    Основная цель — заставить робота перемещаться в заданную позицию с помощью PID‑регулятора,
    настроенного на управление линейным и угловым положением.
    PID‑контроллер используется для регулирования скорости и ориентации робота так,
    чтобы он достигал требуемой позиции.

    \textbf{Линейно-квадратичный регулятор (LQR)} основан на минимизации квадратичного функционала качества,
    учитывающего отклонение состояния системы от заданного и величину управляющего воздействия.
    В работах~\cite{lqr} и~\cite{lqr2} показано,
    что LQR позволяет обеспечить более устойчивое и оптимальное движение мобильного робота по сравнению с PID,
    особенно при задаче слежения за траекторией.
    LQR учитывает модель динамики системы,
    что повышает точность управления.
    Его основным недостатком является необходимость линеаризации модели и чувствительность к ошибкам параметров.

    \textbf{Комбинированные схемы PID+LQR} используются для объединения простоты PID и оптимальных свойств LQR\@.
    В работе~\cite{pid_lqr} показано, что такой подход позволяет повысить устойчивость системы,
    улучшить качество позиционирования и подавить колебания.
    Однако гибридные контроллеры требуют более сложной настройки и точного согласования параметров.

    В целом классические контроллеры не являются полноценными локальными планировщиками,
    но выполняют важную роль исполнительного уровня, обеспечивая реализацию траекторий,
    полученных более высокими уровнями навигационной системы.

    \subsubsection{Геометрические или реактивные методы}
    Геометрические или реактивные методы локального планирования принимают решения непосредственно
    на основе текущих сенсорных данных и геометрии окружающей среды.
    Они не строят долгосрочные траектории, а выбирают допустимое направление или скорость движения,
    обеспечивающие безопасность в ближайшем будущем.

    \textbf{Метод Dynamic Window Approach (DWA)} был предложен в~\cite{dwa_origin}.
    Он рассматривает пространство допустимых линейных и угловых скоростей робота и выбирает те из них,
    которые достижимы с учётом динамических ограничений и не приводят к столкновениям.
    Далее используется целевая функция, учитывающая приближение к цели,
    скорость движения и расстояние до препятствий.
    В работе~\cite{dwa_new} предложена модификация DWA, ориентированная на динамические препятствия и групповые движения роботов.
    Основным достоинством DWA является высокая скорость работы и практическая применимость,
    однако метод подвержен застреванию в локальных минимумах.

    \textbf{Метод Velocity Obstacles (VO)} был предложен в~\cite{vo_origin} и основан на построении множества скоростей,
    которые приведут к столкновению с препятствием в будущем.
    Эти скорости исключаются из допустимого пространства управления.
    Такой подход позволяет учитывать движение препятствий и прогнозировать возможные столкновения.
    В работе~\cite{vo_new} предложено расширение VO, учитывающее как положение, так и скорости объектов.
    Преимуществом метода является корректная работа в динамических средах, однако он чувствителен к шуму в измерениях.

    \textbf{Метод искусственных потенциальных полей (APF)} был предложен Хатибом в~\cite{apf_orig}.
    Цель моделируется как источник притяжения, а препятствия — как источники отталкивания,
    и робот движется по направлению результирующего градиента.
    В~\cite{apf_new} предложена модификация APF для задач навигации в неизвестной среде с динамическими препятствиями.
    Основными преимуществами APF являются простота и высокая вычислительная эффективность.
    Существенным недостатком является наличие локальных минимумов, из-за которых робот может застревать.

    \textbf{Метод Vector Field Histogram (VFH)} был представлен в~\cite{vfh_orig} и
    основан на построении гистограммы плотности препятствий по данным сенсоров.
    На её основе выбирается направление движения с минимальной опасностью столкновения.
    В работе~\cite{vfh_new} предложены модификации VFH,
    улучшающие точность построения гистограмм и учитывающие динамические препятствия.
    Метод хорошо работает в реальном времени, однако не обеспечивает глобальной оптимальности траектории.

    Геометрические методы отличаются высокой вычислительной эффективностью и
    широко применяются в реальных робототехнических системах.
    Их основным недостатком является локальный характер принятия решений,
    что может приводить к неоптимальному поведению в сложных средах.

    \subsubsection{Оптимизационные или предсказательные методы}

    Оптимизационные методы формулируют задачу локального планирования как задачу оптимального управления,
    в которой необходимо минимизировать функционал качества при наличии динамических и кинематических ограничений.

    \textbf{Model Predictive Control (MPC)} является одним из наиболее распространённых методов этого класса.
    В обзоре~\cite{mpc} показано, что MPC позволяет учитывать динамику робота,
    ограничения на управление и наличие препятствий.
    Алгоритм на каждом шаге решает задачу оптимизации на конечном горизонте прогнозирования,
    обеспечивая высокое качество траекторий.
    Основным недостатком является высокая вычислительная сложность.

    \textbf{Model Predictive Path Integral (MPPI)} представляет собой стохастический вариант MPC,
    основанный на выборке большого числа возможных траекторий и оценке их стоимости~\cite{mppis}.
    В работе~\cite{mppi} MPPI успешно применён для задачи агрессивного автономного вождения.
    Преимуществами MPPI являются возможность работы с нелинейной динамикой и с
    ложными функциями стоимости.
    Недостатком является необходимость больших вычислительных ресурсов.

    \textbf{Алгоритм CCS-MPPI}, предложенный в~\cite{ccsmppi}, объединяет MPPI с методом Covariance Steering и
    позволяет управлять не только средним значением состояния, но и его ковариацией,
    обеспечивая вероятностные гарантии соблюдения ограничений.
    Этот подход особенно эффективен в стохастических средах, но сложен в реализации.

    \textbf{Метод RMPPI}, предложенный в~\cite{rmppi}, направлен на повышение робастности MPPI к ошибкам модели и внешним возмущениям.
    Он сочетает MPPI с трекинг-контроллером и механизмами безопасного распространения номинальной траектории.

    \textbf{Метод Adaptive MPPI}
    предложен в работе~\cite{adaptive_mppi},
    в нём параметры модели и контроллера автоматически настраиваются под текущие условия среды.
    Это позволяет повысить устойчивость алгоритма при неточной модели динамики.

    \textbf{Метод MPPI with Costmap},
    представленный в работе~\cite{mppi_with_map}, использует локальную карту затрат (cost map),
    формируемую на основе априорной карты и данных с камеры,
    что позволяет интегрировать глобальную информацию в локальное планирование.

    Оптимизационные методы обеспечивают наивысшее качество траекторий и наилучшее учёт динамики робота и среды,
    однако их применение ограничено высокими вычислительными затратами и сложностью реализации.


    %\makecell[c]{}

    \pagebreak

    \subsection{Сравнительный анализ и выбор алгоритмов}

    Для систематизации полученных данных и выбора оптимального стека технологий был проведен сравнительный анализ алгоритмов.
    В таблицах~\ref{table:global_algorithms}, ~\ref{table:local_algorithms} представлены ключевые характеристики как глобальных,
    так и локальных планировщиков.

    Для оценки эффективности используются следующие обозначения:
    \begin{itemize}
        \item[«$+$»] — высокий показатель (высокая скорость работы, эффективная работа с динамикой, высокая точность);
        \item[«$+-$»] — средний показатель (зависит от настроек, средние вычислительные затраты);
        \item[«$-$»] — низкий показатель (медленная работа, неэффективность в динамике, риск локальных минимумов).
    \end{itemize}

    \begin{table}[ht]
        \caption{Сравнение алгоритмов глобального планирования}
        \label{table:global_algorithms}
        \footnotesize
        \centering
        \begin{tabularx}{\textwidth}{|c|c|c|c|c|}
            \hline
            \makecell[c]{Класс\\алгоритм} &
            \makecell[c]{Работа в\\динамической\\среде} &
            \makecell[c]{Вычислительная\\сложность} &
            \makecell[c]{Основные\\преимущества} &
            \makecell[c]{Основные\\недостатки}\\
            \hline
            \makecell[c]{Графовые\\\textbf{Dijkstra}} &
            $-$ & $+$ & \makecell[c]{Гарантированная\\оптимальность,\\простота} &
            \makecell[c]{Полный пересчёт\\при изменениях}\\
            \hline
            \makecell[c]{Графовые\\\textbf{A*}} &
            $-$ & $+-$ &
            \makecell[c]{Быстрее Дейкстры,\\эвристический поиск} & \makecell[c]{Зависит от эвристики}\\
            \hline
            \makecell[c]{Графовые\\\textbf{D*}} &
            $+$ & $+-$ &
            \makecell[c]{Инкрементальное\\перепланирование} & \makecell[c]{Сложная реализация}\\
            \hline
            \makecell[c]{Графовые\\\textbf{D* Lite}} &
            $+$ & $+-$ &
            \makecell[c]{Эффективное\\обновление пути} & \makecell[c]{Требует памяти}\\
            \hline
            \makecell[c]{Графовые\\\textbf{LPA*}} &
            $+$ & $+-$ &
            \makecell[c]{Повторное\\использование\\поиска} & \makecell[c]{Медленнее\\D* Lite}\\
            \hline
            \makecell[c]{Графовые\\\textbf{ARA*}} &
            $+-$ & $+-$ &
            \makecell[c]{Быстрое\\первое решение} & \makecell[c]{Временно\\субоптимален}\\
            \hline
            \makecell[c]{Семплинг\\\textbf{RRT}} &
            $+-$ & $-$ &
            \makecell[c]{Быстро находит путь} & \makecell[c]{Плохое качество пути}\\
            \hline
            \makecell[c]{Семплинг\\\textbf{RRT*}} &
            $+-$ & $+$ &
            \makecell[c]{Оптимальные\\траектории} & \makecell[c]{Медленная сходимость}\\
            \hline
            \makecell[c]{Семплинг\\\textbf{BIT*}} &
            $+-$ & $+-$ &
            \makecell[c]{Эвристическое\\ускорение} & \makecell[c]{Сложная настройка}\\
            \hline
            \makecell[c]{Семплинг\\\textbf{RABIT*}} &
            $+-$ & $+-$ &
            \makecell[c]{Быстрая оптимизация} & \makecell[c]{Зависит от локального\\оптимизатора}\\
            \hline
            \makecell[c]{Семплинг\\\textbf{Risk-DTRRT}} &
            $+$ & $+$ &
            \makecell[c]{Учет неопределённости} & \makecell[c]{Сложность модели}\\
            \hline
            \makecell[c]{Бионические\\\textbf{GA}} &
            $+-$ & $+$ &
            \makecell[c]{Гибкость, адаптация} & \makecell[c]{Медленная сходимость}\\
            \hline
            \makecell[c]{Бионические\\\textbf{ACO}} &
            $+-$ & $+-$ &
            \makecell[c]{Распределённый\\поиск} & \makecell[c]{Чувствителен к параметрам}\\
            \hline
            \makecell[c]{Бионические\\\textbf{ABC}} &
            $+-$ & $+-$ &
            \makecell[c]{Простота реализации} & \makecell[c]{Нестабильность}\\
            \hline
            \makecell[c]{Бионические\\\textbf{PSO}} &
            $+-$ & $+-$ &
            \makecell[c]{Быстро сходится} & \makecell[c]{Локальные минимумы}\\
            \hline
        \end{tabularx}

    \end{table}

    \begin{table}[ht]
        \caption{Сравнение алгоритмов локального планирования}
        \label{table:local_algorithms}
        \footnotesize
        \centering
        \begin{tabularx}{\textwidth}{|c|c|c|c|c|}
            \hline
            \makecell[c]{Класс\\алгоритм} &
            \makecell[c]{Работа в\\динамической\\среде} &
            \makecell[c]{Вычислительная\\сложность} &
            \makecell[c]{Основные\\преимущества} &
            \makecell[c]{Основные\\недостатки}\\
            \hline
            \makecell[c]{Классические\\контроллеры\\\textbf{PID}} &
            $-$ & $-$ & \makecell[c]{Простота, \\высокая частота\\управления} &
            \makecell[c]{Нет планирования,\\нет обхода препятствий}\\
            \hline
            \makecell[c]{Классические\\контроллеры\\\textbf{LQR}} &
            $+-$ & $+-$ & \makecell[c]{Оптимальность, \\устойчивость} &
            \makecell[c]{Требует линеаризации}\\
            \hline
            \makecell[c]{Классические\\контроллеры\\\textbf{PID+LQR}} &
            $+-$ & $+-$ & \makecell[c]{Повышенная \\устойчивость и \\точность} &
            \makecell[c]{Сложность настройки}\\
            \hline
            \makecell[c]{Геометрические\\\textbf{DWA}} &
            $+$ & $-$ & \makecell[c]{Реактивность, \\практическая\\применимость} &
            \makecell[c]{Локальные минимумы}\\
            \hline
            \makecell[c]{Геометрические\\\textbf{VO}} &
            $+$ & $+-$ & \makecell[c]{Учёт движущихся\\препятствий} &
            \makecell[c]{Чувствительность\\к шуму}\\
            \hline
            \makecell[c]{Геометрические\\\textbf{APF}} &
            $+-$ & $-$ & \makecell[c]{Простота,\\высокая скорость} &
            \makecell[c]{Локальные минимумы}\\
            \hline
            \makecell[c]{Геометрические\\\textbf{VFH}} &
            $+$ & $-$ & \makecell[c]{Эффективная\\работа с сенсорами} &
            \makecell[c]{Нет глобальной\\оптимальности}\\
            \hline
            \makecell[c]{Оптимизационные\\\textbf{MPC}} &
            $+$ & $+$ & \makecell[c]{Учёт ограничений\\и динамики} &
            \makecell[c]{Большие\\вычислительные затраты}\\
            \hline
            \makecell[c]{Оптимизационные\\\textbf{MPPI}} &
            $+$ & $+$ & \makecell[c]{Качественные\\траектории} &
            \makecell[c]{Требует параллелизма}\\
            \hline
            \makecell[c]{Оптимизационные\\\textbf{CCS-MPPI}} &
            $+$  & $+$  & \makecell[c]{Вероятностные\\гарантии\\безопасности} &
            \makecell[c]{Сложная реализация}\\
            \hline
            \makecell[c]{Оптимизационные\\\textbf{RMPPI}} &
            $+$ & $+$  & \makecell[c]{Робастность к\\ошибкам модели} &
            \makecell[c]{Высокая\\вычислительная цена}\\
            \hline
            \makecell[c]{Оптимизационные\\\textbf{Adaptive MPPI}} &
            $+$ & $+$ & \makecell[c]{Адаптация к\\неопределённости} &
            \makecell[c]{Сложность настройки}\\
            \hline
            \makecell[c]{Оптимизационные\\\textbf{MPPI + Costmap}} &
            $+$  & $+$  & \makecell[c]{Интеграция\\глобальной\\информации} &
            \makecell[c]{Большая\\вычислительная нагрузка}\\
            \hline


        \end{tabularx}

    \end{table}

    \subsubsection{Обоснование выбора алгоритма D* Lite}
    На основе проведенного анализа для решения задачи навигации в динамических средах выбран алгоритм D* Lite.

    В условиях непрерывного обновления карты сенсорными данными использование классического
    A* становится неэффективным: необходимость полной перестройки графа при
    каждом изменении среды создает критические вычислительные задержки.
    Алгоритм D* Lite нивелирует этот недостаток за счет механизма инкрементального поиска,
    при котором пересчитываются только участки пути, непосредственно затронутые изменениями.
    Это позволяет достичь оптимального баланса между глобальной эффективностью маршрута и
    высокой скоростью реакции, характерной для локальных планировщиков.
    Кроме того, относительная алгоритмическая простота D* Lite упрощает его
    интеграцию в гибридные навигационные системы и облегчает процесс отладочного тестирования.

   \subsubsection{Выводы по главе}

Проведенный анализ подтверждает отсутствие универсального алгоритма, способного одновременно обеспечивать оптимальный глобальный охват и мгновенное избегание препятствий. Глобальные планировщики гарантируют достижение целевой точки и отсутствие тупиковых ситуаций, однако обладают высокой вычислительной инертностью. Напротив, локальные методы (такие как DWA или MPC) обеспечивают оперативную безопасность движения, но без глобального ориентира могут приводить к зацикливанию робота в сложных пространственных структурах.



Наиболее эффективным решением признана гибридная архитектура, в которой за инкрементальный пересчет маршрута отвечает алгоритм $D^*$ Lite, а за безопасную отработку траектории в режиме реального времени — локальный планировщик. Такое сочетание минимизирует время простоя робота при изменении условий среды, обеспечивая баланс между глобальной оптимальностью пути и высокой скоростью реакции на динамические помехи.
    \newpage


    \section{Экспериментальное исследование и анализ результатов}

    В данной главе представлены результаты сравнительного тестирования алгоритмов $A^*$ и $D^*$ Lite.
    Целью экспериментов является количественная оценка преимуществ инкрементального подхода при работе в динамических средах большой площади.

    \subsection{Методика и условия проведения эксперимента}

    Тестирование проводилось на двумерной карте \textbf{Aurora} ($1024 \times 768$ узлов) из набора MovingAI.
    Данная локация представляет собой сложный граф, содержащий почти $800\,000$ вершин,
    что позволяет эффективно выявить и проанализировать вычислительные задержки рассматриваемых алгоритмов в условиях,
    приближенных к реальным эксплуатационным масштабам.


    Экспериментальное исследование было разделено на два последовательных этапа.
    На этапе статического тестирования выполнялось 100 независимых запусков построения первичного
    маршрута между случайными парами точек.
    Это позволило собрать статистически значимые данные о базовой производительности алгоритмов при
    инициализации в условиях полностью известной и неизменной среды.

    Второй этап включал динамическое тестирование,
    в ходе которого имитировалось движение робота по ранее проложенному пути.
    На каждом шаге перемещения одно случайное препятствие меняло свои координаты,
    что с высокой вероятностью вызывало коллизию с текущей траекторией и требовало немедленного перепланирования.
    Данный сценарий позволил оценить ключевое преимущество инкрементальных вычислений —
    скорость адаптации маршрута к локальным изменениям графа без полной остановки системы навигации.

    \subsection{Анализ производительности в статической среде}

    На первом этапе тестирования проводилось сравнение времени построения первичного маршрута.
    Результаты 100 независимых запусков на карте Aurora позволяют сделать вывод о
    базовой вычислительной сложности алгоритмов.

    \begin{table}[ht]
        \caption{Сравнительная таблица производительности алгоритмов в статической среде}
        \label{table:static_results}
        \footnotesize
        \centering
        % Всего 4 столбца. X - растягиваемый столбец, c - центрированные по контенту
        \begin{tabularx}{\textwidth}{|X|c|c|c|}
            \hline
            \textbf{Алгоритм} & \textbf{Среднее время (мс)} & \textbf{Мин. время (мс)} & \textbf{Макс. время (мс)} \\
            \hline
            $A^*$ & 38,41 & 1,76 & 146,46 \\
            \hline
            $D^*$ Lite & 163,54 & 27,82 & 557,76 \\
            \hline
        \end{tabularx}
    \end{table}



    Анализ данных показывает, что в статической среде A∗ превосходит D∗ Lite в среднем в 4,25 раза.
    Данная разница является ожидаемой и обусловлена архитектурными особенностями инкрементальных методов:
    необходимостью расчета и хранения двух оценок стоимости для каждого узла графа,
    а также поддержанием более сложной структуры приоритетной очереди с составными ключами.
    A∗ в данном сценарии является более легковесным решением,
    так как не несет вычислительной нагрузки по подготовке структур данных к будущим изменениям среды.

    \subsection{Результаты тестирования в динамической среде}

    Вторая серия экспериментов была направлена на оценку эффективности алгоритмов при возникновении коллизий на уже построенном маршруте. Было проведено 100 независимых тестов в условиях динамики.

    Методика теста заключалась в следующем: на каждом шаге симуляции случайное препятствие на карте Aurora меняло свои координаты. Если новое положение препятствия перекрывало текущий путь робота, фиксировалось время, затраченное алгоритмом на поиск обходного пути.

    Для количественной оценки эффективности в условиях изменяющейся среды была сформирована сводная таблица статистических показателей (см. табл.~\ref{table:dynamic_results}). Основными метриками выступали временные затраты на одну операцию перепланирования и стабильность этих затрат при многократном повторении сценария.

    \begin{table}[ht]
        \caption{Сравнительная таблица производительности алгоритмов в динамической среде}
        \label{table:dynamic_results}
        \footnotesize
        \centering
        \begin{tabularx}{\textwidth}{|X|c|c|}
            \hline
            \textbf{Показатель} & \textbf{Алгоритм $A^*$} & \textbf{Алгоритм $D^*$ Lite} \\
            \hline
            Среднее время (мс) & 35,12 & 1,18 \\
            \hline
            Среднеквадратичное отклонение ($\sigma$) & 22,45 & 0,84 \\
            \hline
            Коэффициент вариации & 63,9\% & 71,1\% \\
            \hline
        \end{tabularx}
    \end{table}


    В данной таблице среднее время отражает среднюю задержку на пересчет маршрута при обнаружении нового препятствия. Существенный разрыв в значениях (1,18~мс у $D^*$ Lite против 35,12~мс у $A^*$) подтверждает эффективность инкрементального поиска, который исключает повторную обработку не затронутых изменениями участков графа.

    Среднеквадратичное отклонение ($\sigma$) и коэффициент вариации позволяют оценить предсказуемость работы системы. Высокое значение $\sigma$ для алгоритма $A^*$ свидетельствует о том, что время вычислений сильно зависит от расположения препятствия, в то время как крайне низкое абсолютное отклонение у $D^*$ Lite гарантирует стабильную работу планировщика в режиме реального времени.

    \subsection{Анализ и интерпретация результатов}

    Сравнение данных из таблиц~\ref{table:static_results} и~\ref{table:dynamic_results} позволяет сделать вывод о существенном преимуществе инкрементального подхода в условиях изменяющейся среды. В динамических сценариях алгоритм $D^*$ Lite демонстрирует среднее время отклика в \textbf{29,7 раза} быстрее, чем $A^*$. Столь значительный выигрыш в производительности объясняется тем, что $D^*$ Lite пересчитывает только те участки графа, которые непосредственно затронуты изменениями, в то время как $A^*$ при каждом обнаружении препятствия на пути вынужден заново проводить поиск по всей карте.


    Важным аспектом является предсказуемость времени работы системы. Для алгоритма $D^*$ Lite среднеквадратичное отклонение составило всего \textbf{0,84~мс}, что указывает на стабильность вычислений. Для мобильного робота это гарантирует выполнение задач в рамках жестких временных ограничений управляющего цикла (Real-time constraints). В то же время алгоритм $A^*$ показывает большой разброс времени выполнения со значением $\sigma = \textbf{22,45~мс}$. Задержки у $A^*$ (от 1 до 118~мс) сильно зависят от того, насколько далеко от робота возникло препятствие, что затрудняет прогнозирование нагрузки на процессор.

    При анализе работы в статических средах зафиксировано преимущество алгоритма $A^*$, который выполняет первичный поиск в среднем в 4,25 раза быстрее. Это связано с тем, что $A^*$ использует более простые структуры данных и не тратит ресурсы на хранение дополнительной информации, необходимой для быстрого перепланирования. Таким образом, $A^*$ эффективнее для разового построения маршрута, однако его производительность падает при необходимости частого обновления пути.

    На крупномасштабных картах, таких как Aurora ($1024 \times 768$), использование $D^*$ Lite позволяет существенно снизить общую нагрузку на процессор. Освободившиеся ресурсы могут быть использованы для других задач: высокочастотного локального планирования, обработки данных с сенсоров и распознавания объектов. Результаты 100 контрольных тестов полностью подтверждают, что в динамических средах инкрементальные методы поиска являются наиболее эффективным решением.
    \newpage


    \section*{Заключение}
    \addcontentsline{toc}{section}{Заключение}

    В ходе работы был проведен сравнительный анализ алгоритмов $A^*$ и $D^*$ Lite в динамических средах. На основе 100 контрольных тестов на карте Aurora ($1024 \times 768$) получены следующие результаты:

    \begin{enumerate}
        \item В статических условиях алгоритм $A^*$ выполняет поиск в 4,25 раза быстрее за счет более простых структур данных.
        \item В динамической среде алгоритм $D^*$ Lite превосходит $A^*$ по скорости перепланирования в 29,7 раза со средним временем отклика 1,18~мс.
        \item Анализ стабильности показал, что среднеквадратичное отклонение у $D^*$ Lite составляет 0,84~мс, тогда как у $A^*$ — 22,45~мс. Это гарантирует предсказуемую работу системы в реальном времени.
    \end{enumerate}

    Таким образом, подтверждено, что $D^*$ Lite является оптимальным решением для навигации в изменяющихся условиях, так как обеспечивает мгновенную реакцию на препятствия и снижает общую нагрузку на бортовой компьютер.
    \newpage
    \printbibliography[heading=bibintoc]

\end{document}
